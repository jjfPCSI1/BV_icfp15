\section{Introduction}

Marc de Falco a propos� durant l'�t� aux membres de l'UPS de participer au 
tournoi de programmation de l'ICFP\footnote{International Conference on 
Functional Programming} (voir \url{icfpcontest.org}) qui s'est tenu sur 
trois jours du vendredi 7 ao�t 2015 (14h) au lundi 10 ao�t 2015 (14h). � son 
appel, une �quipe (TaupeGoons) s'est form�e pour relever le d�fi. Elle �tait constitu�e 
de quatre membres:

\begin{itemize}
	\item	Marc de Falco, le GO du groupe qui a tout orchestr� et a abattu la 
	majorit� du travail;
	
	\item	Jean-Julien Fleck, qui s'est occup� de l'optimisation des 
	param�tres par algorithme g�n�tique ;
	
	\item	Laurent Jospin, qui malheureusement a d� d�clarer forfait suite � 
	des soucis informatiques ;
	
	\item	Lo�c Pottier, qui a d�velopper son propre moteur de r�solution et 
	a fini par former sa propre �quipe (GaupeToons), d'abord pour le tester, 
	puis pour essayer de l'am�liorer puisque les r�sultats �taient plut�t 
	bons.
	
\end{itemize}

Le pr�sent article vise � d�crire le probl�me pos� pour le tournoi ainsi que 
la mani�re dont l'�quipe TaupeGoons s'est ing�ni� � le r�soudre. Tout le code 
informatique utilis� peut �tre retrouv� � l'adresse

\begin{center}
\url{https://github.com/MarcdeFalco/icfp15}
\end{center}
