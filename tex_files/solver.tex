\section{Solution choisie}
\subsection{Hachage des positions}
Afin de pouvoir assurer qu'aucune r�p�tition de position a lieu il ne suffit pas de prendre 
en compte des couples $(M,r)$ o� $M$ est la position du pivot et $r$ la rotation effectu�s sur la 
pi�ce.

En effet, certaines pi�ces sont invariantes par certaines rotations et la seule donn�e de $r$
ne permet pas de le prendre en compte. On consid�re pour cela son groupe d'invariants.

Soit $p$ une pi�ce on note $r_p$ la rotation d'un angle $\frac{\pi}{6}$ autour de son pivot.

On note $R(p)$ le sous-groupe des isom�tries du plan engendr�es par $r_p$ et $I(p)$ 
le sous-groupe de $R(p)$ constitu� des �l�ments laissant $p$ invariante.

Pour une position donn�e du pivot, on a donc autant de possibilit�s pour la pi�ce que
d'�l�ments dans $G_r(p) = R(p) / I(p)$ (tous ces groupes sont ab�liens). 

Comme $R(p) \sim \Z/6\Z$ on a quatre cas pour $G_r(p)$ :
\begin{itemize}
\item $G_r(p) \sim \Z/6\Z$. Exemple :

    \begin{tikzpicture}
    \begin{scope}
        \hexgrid{2}{2}
        \unit{0/1,1/1}
        \pivot{1}{1}
    \end{scope}

    \node at (3,-0.5) {\Large $\circlearrowright$};

    \begin{scope}[xshift=4cm]
        \hexgrid{2}{2}
        \unit{1/0,1/1}
        \pivot{1}{1}
    \end{scope}

    \node at (7,-0.5) {\Large $\circlearrowright$};

    \begin{scope}[xshift=8cm]
        \hexgrid{2}{2}
        \unit{2/0,1/1}
        \pivot{1}{1}
    \end{scope}
    \end{tikzpicture}


    \begin{tikzpicture}
    \node at (-.75,-0.5) {\Large $\circlearrowright$};
    \begin{scope}
        \hexgrid{2}{2}
        \unit{2/1,1/1}
        \pivot{1}{1}
    \end{scope}

    \node at (3,-0.5) {\Large $\circlearrowright$};

    \begin{scope}[xshift=4cm]
        \hexgrid{2}{2}
        \unit{2/2,1/1}
        \pivot{1}{1}
    \end{scope}

    \node at (7,-0.5) {\Large $\circlearrowright$};

    \begin{scope}[xshift=8cm]
        \hexgrid{2}{2}
        \unit{1/2,1/1}
        \pivot{1}{1}
    \end{scope}
    \end{tikzpicture}

\item $G_r(p) \sim \Z/3\Z$. Exemple :

    \begin{tikzpicture}
    \begin{scope}
        \hexgrid{2}{2}
        \unit{0/1,1/1,2/1}
        \pivot{1}{1}
    \end{scope}

    \node at (3,-0.5) {\Large $\circlearrowright$};

    \begin{scope}[xshift=4cm]
        \hexgrid{2}{2}
        \unit{1/0,1/1,2/2}
        \pivot{1}{1}
    \end{scope}

    \node at (7,-0.5) {\Large $\circlearrowright$};

    \begin{scope}[xshift=8cm]
        \hexgrid{2}{2}
        \unit{2/0,1/1,1/2}
        \pivot{1}{1}
    \end{scope}
    \end{tikzpicture}

\item $G_r(p) \sim \Z/2\Z$. Exemple :

    \begin{tikzpicture}

    \begin{scope}
        \hexgrid{2}{2}
        \unit{1/0,2/1,1/2}
        \pivot{1}{1}
    \end{scope}

    \node at (3,-0.5) {\Large $\circlearrowright$};

    \begin{scope}[xshift=4cm]
        \hexgrid{2}{2}
        \unit{0/1,2/0,2/2}
        \pivot{1}{1}
    \end{scope}

    \end{tikzpicture}

\item $G_r(p) \sim \mathbf{0}$. Exemples :

    \begin{tikzpicture}
        \hexcell{0}{0}
        \pivot{0}{0}
    \end{tikzpicture} 
    \quad
    \begin{tikzpicture}
        \hexgrid{2}{2}
        \unit{1/0,2/0,0/1,1/1,2/1,1/2,2/2}
        \pivot{1}{1}
    \end{tikzpicture} 
\end{itemize}

On peut alors repr�senter uniquement la position d'une pi�ce � sym�trie pr�s par un couple 
$(M,r)$ o� $M$ est la position du pivot et $r \in G_r(p)$.

En pratique, on identifie $r$ � l'�l�ment $k \in \{0,1,...,5\}$ tel que $r$ soit 
associ� � $\overline{k}$ par les isomorphismes ci-dessus.

On est donc ramen� � un triplet d'entier par position. Comme le pivot peut sortir
de la zone de jeu, il faut border le tableau pour pouvoir tenir compte de ces positions du pivot.

Un calcul rapide sur les pi�ces disponibles permet de d�terminer une valeur $b$ telle que le pivot 
dans une zone de jeu de dimension $w \times h$ ait des coordonn�es dans $[|-b;w+b-1|] \times
[|-b;h+b-1|]$. 

On peut alors cr�er un tableau de bool�ens � trois dimensions $V$ tel que $V_{x,y,k}$
indique si la position o� le pivot est en $(x,y)$ et la rotation est celle associ�e � $k$.
