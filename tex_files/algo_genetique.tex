\section{Optimisation des param�tres par algorithme g�n�tique}


Afin de s�lectionner la meilleure feuille de l'arbre construit par 
l'algorithme principal, il faut lui attribuer un score pond�r� par un certain 
nombre de param�tres. Ces coefficients de pond�ration ont �t� choisis \ofg{au 
doigt mouill�} dans un premier temps en fonction de l'intuition que l'on avait 
de leur effet potentiel, mais une fois l'algorithme lach� sur un probl�me, il 
n'est pas certain que le choix soit optimal. On a donc d�cid� d'utiliser un 
algorithme g�n�tique pour essayer de trouver rapidement un jeu de pond�rations 
qui puissent faire mieux que le jeu par d�faut.

\subsection{Id�e de l'algorithme g�n�tique}

Le principe de l'algorithme g�n�tique est de faire \ofg{s'affronter} 
diff�rents jeux de param�tres pour pouvoir les classer en fonction d'un 
certain crit�re (ici, ce sera le score total sur l'ensemble des probl�mes 
soumis). Une fois les diff�rents candidats rang�s par ordre d'efficacit�s, on 
s�lectionne les meilleurs et on les \ofg{reproduit} entre eux en m�langeant 
leurs caract�ristiques principales pour former un certains nombre de 
programmes \ofg{fils}. Les parents et les enfants s'affrontent alors � nouveau 
sur le jeu de probl�me et on res�lectionne les meilleurs du cheptel pour la 
reproduction afin de produire une nouvelle g�n�ration. Au bout d'un certains 
nombre de g�n�rations (les effets commencent � se faire sentir � partir de la 
troisi�me), la \ofg{s�lection naturelle} fait ressortir des jeux de param�tres 
qui peuvent notablement am�liorer le score du jeu initial.

\subsection{S�lection initiale des candidats}

\subsection{Organisation du tournoi}

\subsection{Reproduction des meilleurs}

\subsection{Importance des mutations}


